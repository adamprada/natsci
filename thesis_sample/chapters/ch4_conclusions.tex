\chapter{Conclusions and further work}
%\textcolor{red}{
%Most chemical dynamics methods for condensed phase simulations include solvent explicitly. It is the most accurate, but also the most computationally demanding approach. An alternative approach involves replacing most of the surrounding molecules with a bath, which may be able to capture the essence of the interaction with the environment at a smaller computational cost.\supercite{Nitzan2013} The resulting equations of motion are termed dissipative dynamics and are routinely used in classical molecular dynamics. However, the quantum treatment of dissipative systems is much less developed.\supercite{Weiss2012}
%}
%
Hierarchical equations of motion (HEOM) are an exact description of the propagation of a quantum system coupled to a bath due to Tanimura. They are a versatile tool for computing dissipative quantum dynamics.\supercite{Tanimura1990a,Tanimura1991a,Tanimura1992,Tanimura2006a}

In this work, a computer program capable of a range of HEOM simulations was developed. Firstly, a reduced version of HEOM was implemented, omitting the Matsubara frequency terms in the bath time autocorrelation function. Its behaviour was investigated under a range of conditions, including a breakdown for low temperatures. At such conditions, the position time autocorrelation functions were accurate only for very short times, but showing non-physical beats at longer times. In addition, the stability of the wave packets during equilibration was impaired, leading to artificial drifting or changes in shapes.

Taking a Wigner transform of the quantum HEOM followed by a high temperature limit leads to fully classical HEOM.\supercite{Tanimura1991a} These were implemented using discretised position-momentum phase space representations of Wigner functions and a DVR\=/based first derivative matrices. Results obtained from these simulations were compared to the quantum results.

To correct for the breakdown of the reduced quantum HEOM, a full version containing the Matsubara terms of the bath correlation function\supercite{Shi2009a} was implemented. It was shown how inclusion of these terms leads to a reduction and complete removal of the beats in the position time autocorrelation functions as well as to the mitigation of the instabilities during equilibration.

To test our implementation of HEOM, we have set out to reproduce Morse oscillator spectra at different bath cut-off frequencies which were originally presented by Sakurai and Tanimura.\supercite{Sakurai2011} Our calculations matched the published results closely.

Leading up to further work, we began an investigation of a hydrated oxonium ion OH bond potential from an article by Yu and Bowman.\supercite{Yu2019} They have shown that this potential is poorly tackled by most currently used quantum dynamics methods. Raz Benson calculated white-noise thermostatted ring-polymer molecular dynamics (TRPMD) time correlation functions for a quartic fit to this potential with varying degrees of anharmonicity. TRPMD showed an increasingly large blue shift with increasing anharmonicity. This is a well known shortcoming of all path integral based quantum dynamics methods. One possible hypothesis was that the blue shift might be simply due to the omission of quantum coherence in this family of methods. In such case one would expect that HEOM, which describe coupling to a bath and consequent decoherence exactly, should reproduce at least some of the blue shift. Our calculations have shown that this is not the case. HEOM leads to a blue shift if the renormalisation potential is included, which, however, is a mechanical consequence of the set-up. If the renormalisation potential is set to zero, then a small red shift is observed instead. The origin of this shift could be purely mechanical or related to the coherence but in the light of this observation it seems unlikely that quantum coherence would be an major factor in the blue shift of TRPMD. This result is yet to be compared to related methods like centroid molecular dynamics (CMD)\supercite{Cao1994h, Jang1999a} and coloured noise TRPMD (TRPMD+GLE).\supercite{Rossi2018}

In the following work, we would like to implement TRPMD with an explicit bath. This would give us access to both exact quantum results for a dissipative system, where quantum coherence can be quenched, as well as to an entirely equivalent system-bath ring-polymer set-up. Such a comparison has never been done and could bring new insight into the approximations inherent to ring-polymer methods as well as the role of quantum coherence in condensed phase dynamics.

A slightly more distant goal, which follows from our preliminary classical HEOM investigations, is implementing Matsubara dynamics with an explicit bath. Unlike in RPMD, which has been shown to work with an explicit set of harmonic oscillators,\supercite{Craig2005b} in Matsubara dynamics it is impractical to do the same due to a severe phase problem.\supercite{Hele2015} Therefore, we seek an alternative route by implementing a bath for Matsubara dynamics using HEOM. Given that Matsubara dynamics is the least approximative of all ring-polymer based methods, such a description would help us to push the understanding of condensed phased quantum dynamics even further.
