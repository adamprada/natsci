\chapter{Introduction}
Advances of molecular dynamics have made it possible to explain many chemical and physical phenomena from first principles.\supercite{Tuckerman2010} For many purposes, atomic nuclei can be treated classically, but over the past few decades the shortcomings of this approximation have become increasingly apparent resulting in the emergence of a new field of quantum chemical/nuclear dynamics.\supercite{Markland2018b}
	
For quantum nuclear dynamics in condensed phase, where the coherence effects can be neglected, a range of path integral based methods have been developed. Using Feynman path integrals, one can obtain an isomorphism which maps the quantum system onto an extended classical system, called a ring polymer.\supercite{Feynman2010} These polymers exactly represent quantum statistics, and classical dynamics is routinely used to sample thermal averages of static variables in a method called path integral molecular dynamics (PIMD).\supercite{Tuckerman2010} However, all the dynamics present is considered to be a mathematical tool, rather than a representation of reality. To calculate dynamical variables, like time correlation functions, a range of PIMD-based methods have been proposed, including centroid molecular dynamics (CMD),\supercite{Cao1994h, Jang1999a} ring polymer molecular dynamics (RPMD),\supercite{Craig2004} linearised semi-classical initial value representation (LSC-IVR)\supercite{Wang1998} and related methods (PA-CMD, TRPMD, etc.).\supercite{Markland2018b} All of these were originally \emph{ad hoc} propositions that were shown to yield correct answers in various limits. In 2015 a new theory has been proposed, called the Matsubara dynamics, which gives a quantum Boltzmann conserving classical component of the exact quantum dynamics, neglecting only real-time quantum coherence. It was shown that all the previously mentioned methods are approximations to this more exact treatment.\supercite{Hele2015,Hele2015a}

The most explicit way of accounting for the environment in condensed phase is by directly including the solvent molecules and converging the simulation with respect to the size of the box. This will naturally yield the most accurate results, but becomes impractical for larger systems or more computationally expensive methods.\supercite{Frenkel1996} An alternative is treating the surroundings as a non-specific bath, rather than explicit molecules, which is typically modelled as an ensemble of harmonic oscillators. We can push the abstraction even further and generalise the effect of the bath into new equations of motion only for the system of interest. Termed dissipative dynamics, for classical particles such treatment is described using the (generalised) Langevin equation or in case of probability distributions using the Fokker-Planck equation.\supercite{Weiss2012, Nitzan2013} However, there is not a simple way of deriving equivalent quantum equations, which poses a severe limitation given the computational cost of quantum methods.

If the bath is considered to have an effectively infinite temperature, it is possible to use the stochastic Liouville equation\supercite{Kubo1963} and this approach has seen many successes including the prediction of NMR and M\"ossbauer spectra and of dielectric relaxation.\supercite{Tanimura2006a} However, in this treatment only fluctuations are present and no dissipation. This means that particles in a continuous coordinate, like a potential well, can never reach thermal equilibrium, which limits its applicability.\supercite{Tanimura2006a} To overcome this drawback, Tanimura proposed a new approach called the hierarchical equations of motion (HEOM).\supercite{Tanimura1990a} These were based on his previous work with Kubo\supercite{Tanimura1989, Tanimura1989b} on stochastic and dynamic approaches to quantum decoherence and further developed in collaboration Wolynes.\supercite{Tanimura1991a,Tanimura1992} HEOM can be used to derive equations of motion for a reduced density matrix of a system in a dissipative bath, which in appropriate limits are exact. These result in a theoretically infinite hierarchy of coupled differential equations in an infinite-dimensional space. However, as shown in the literature, this hierarchy can be effectively truncated to give a viable computational method.\supercite{Tanimura1991a,Tanimura1992,Tanimura2006a} HEOM were used to calculate a wide range of phenomena like vibrational spectra and electronic spectra, including multi-dimensional and non-linear spectroscopy,\supercite{Sakurai2011} reaction rates\supercite{Chen2009a} and even energy transfer in light harvesting proteins.\supercite{Kreisbeck2011}
	
The aim of this work was to implement the hierarchical equations and to understand their behaviour under a range of conditions. This was to be carried out both in the high-temperature approximation, when Matsubara terms in the bath time autocorrelation function are neglected, as well as in the full low-temperature HEOM treatment, where a more complicated hierarchy of differential equations has to be solved. In addition, HEOM can be Wigner transformed into the position-momentum phase space, which at appropriate limits gives fully classical HEOM. The ability to calculate the exact behaviour of the dissipative system will be an invaluable tool in future studies of the shortcomings of the Matsubara-dynamics based methods including the role of a bath and of quantum coherence.
\newpage
In the next chapter we recapitulate the background theory for quantum dissipative dynamics and guide the reader through the crucial steps of the HEOM derivation. In Chapter 3 we present the results of our computational implementation of the HEOM and discuss their importance. We conclude by summarising our findings and laying down propositions for future work in Chapter 4.